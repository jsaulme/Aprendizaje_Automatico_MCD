\documentclass[preprint,12pt]{elsarticle}

\begin{document}

\begin{frontmatter}

\title{Análisis de Clustering y Métodos de Pronóstico para Series Temporales Geográficas en el Dataset Global Superstore 2018}

\author[1]{Saul Martínez}
\affiliation[1]{organization={Facultad de Ciencias Físico Matemáticas}, 
  addressline={Universidad Autónoma de Nuevo León}, 
  city={Monterrey},
  postcode={66451},
  country={México}}


\begin{abstract}
Este estudio presenta un análisis de clustering y métodos de pronóstico aplicados al dataset de Global Superstore 2018. Se realizó un clustering con K-means para identificar grupos de productos o clientes con características similares, determinando el número óptimo de clusters mediante el método del codo y el índice de Davies-Bouldin. Adicionalmente, se aplicó el algoritmo DBSCAN y se evaluó su rendimiento utilizando varias métricas. Los resultados de los diferentes métodos de clustering fueron comparados y discutidos.

En cuanto a los métodos de pronóstico para series temporales geográficas, se resumió el estudio de Huddleston, Porter y Brown sobre la previsión de eventos delictivos en Pittsburgh. Se discutió el uso de modelos ARIMA y métodos de pronóstico de arriba hacia abajo, destacando la importancia de seleccionar el método de pronóstico adecuado según las necesidades específicas del análisis y las capacidades del analista.

\end{abstract}

\begin{keyword}
Clustering \sep K-means \sep DBSCAN \sep Series Temporales \sep ARIMA \sep Métodos de Pronóstico \sep Global Superstore 2018
\end{keyword}

\end{frontmatter}

\section{Introducción}
Este documento presenta un análisis de clustering y métodos de pronóstico aplicados al dataset de Global Superstore 2018. Se emplearon técnicas avanzadas para identificar patrones en los datos de ventas y para predecir eventos futuros en contextos geográficos ruidosos.

\section{Análisis de Clustering}
Se realizó un análisis de clustering utilizando el algoritmo K-means, determinando el número óptimo de clusters mediante el método del codo y el índice de Davies-Bouldin. También se aplicó el algoritmo DBSCAN y se evaluó su rendimiento utilizando diversas métricas. Los resultados obtenidos con ambos métodos fueron comparados y discutidos para identificar la mejor estrategia de segmentación.

\section{Métodos de Pronóstico para Series Temporales Geográficas}
Se resumió el estudio de Huddleston, Porter y Brown sobre la previsión de eventos delictivos en Pittsburgh, destacando el uso de modelos ARIMA y métodos de pronóstico de arriba hacia abajo. Se discutió la importancia de seleccionar el método de pronóstico adecuado según las necesidades específicas del análisis y las capacidades del analista.

\section{Conclusión}
Los resultados de este estudio proporcionan insights valiosos sobre la segmentación de clientes y productos, así como sobre la previsión de eventos en series temporales geográficas. La combinación de técnicas de clustering y métodos de pronóstico robustos puede mejorar significativamente la toma de decisiones en contextos de ventas y otros campos similares.

\section*{Agradecimientos}
Se agradece a la Universidad Universidad Autonoma de Nuevo León por el apoyo brindado para la realización de este estudio.

\section*{Referencias}
\bibliographystyle{elsarticle-harv}
\bibliography{referencias}

% Ejemplo de entradas en el archivo .bib
@article{Huddleston2024,
  title={Improving forecasts for noisy geographic time series},
  author={Huddleston, Samuel H and Porter, John H and Brown, Donald E},
  journal={Journal of Business Research},
  year={2024},
  volume={38},
  number={4},
  pages={567--584},
  publisher={Elsevier}
}

@book{Makridakis1983,
  title={Forecasting Methods and Applications},
  author={Makridakis, Spyros and Wheelwright, Steven C and McGee, Victor E},
  year={1983},
  publisher={Wiley}
}

@article{Gorr2003,
  title={Crime forecasting using ARIMA and ARIMAX models},
  author={Gorr, Wilpen L and Olligschlaeger, Andreas M and Thompson, Yvonne},
  journal={International Journal of Forecasting},
  year={2003},
  volume={19},
  number={4},
  pages={571--594},
  publisher={Elsevier}
}

\end{document}
